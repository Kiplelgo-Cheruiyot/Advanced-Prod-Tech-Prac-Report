\chapter{Literature Review}
\label{sec:review}
%
\section{Additive Manufacturing}
Additive manufacturing, referred to in short as AM, is the basic principle of generating a model using a three-dimensional Computer-Aided Design (3D CAD)
system and fabricating it directly without the need for process planning. In contrast to other manufacturing processes, AM needs only some basic dimensional details and a small amount of knowledge on how the AM machine works and the materials that are used to build the part.

The basic working principle for AM works is that parts are made by adding material in layers, each layer being a thin cross-section of the part derived from the original CAD data. Layer thickness will affect the final output: the thinner each layer is, the closer the final part will be to the original.

Additive Manufacturing, commonly referred as 3D printing is a computer-based technology. Like many  other manufacturing
technologies, improvements in computing power and reduction in mass storage
costs paved the way for processing the large amounts of data typical of modern 3D
Computer-Aided Design (CAD) models within reasonable time frames. AM takes full advantage of many of the important features of computer technology,
both directly (in the AM machines themselves) and indirectly (within the supporting technology). 

\section{Technologies Associated with AM}
The most common input method for AM technology is to accept a file converted
into the STL file format originally built within a conventional 3D CAD system.
There are, however, other ways in which the STL files can be generated and other
technologies that can be used in conjunction with AM technology. These are:

\subsection{Reverse Engineering Technology}

\subsection{Computer Aided Engineering}

\subsection{Heptic Based CAD}