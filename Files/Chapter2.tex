\chapter{Literature Review}
\label{sec:review}
%
\section{Additive Manufacturing}
Additive manufacturing, referred to in short as AM, is the basic principle of generating a model using a three-dimensional Computer-Aided Design (3D CAD)
system and fabricating it directly without the need for process planning. In contrast to other manufacturing processes, AM needs only some basic dimensional details and a small amount of knowledge on how the AM machine works and the materials that are used to build the part.

The basic working principle for AM works is that parts are made by adding material in layers, each layer being a thin cross-section of the part derived from the original CAD data. Layer thickness will affect the final output: the thinner each layer is, the closer the final part will be to the original.

Additive Manufacturing, commonly referred as 3D printing is a computer-based technology. Like many  other manufacturing
technologies, improvements in computing power and reduction in mass storage
costs paved the way for processing the large amounts of data typical of modern 3D
Computer-Aided Design (CAD) models within reasonable time frames. AM takes full advantage of many of the important features of computer technology,
both directly (in the AM machines themselves) and indirectly (within the supporting technology). 

\section{Technologies Associated with AM}
The most common input method for AM technology is to accept a file converted
into the STL file format originally built within a conventional 3D CAD system.
There are, however, other ways in which the STL files can be generated and other
technologies that can be used in conjunction with AM technology. These are:

\subsection{Reverse Engineering Technology}
Reverse Engineering (RE) is the process of
capturing geometric data from another object, commonly referred to as 3D scanning. These data is initially available “point cloud” form i.e. an unconnected set of points representing the object surfaces. These points need to be connected together
using RE software like Artec, which may also be used to combine point clouds from different scans and to perform other functions like hole-filling and
smoothing.

Engineered objects are scanned using laser-scanning or touch probe technology. Objects that have complex internal features or anatomical
models may make use of Computerized Tomography (CT).

\subsection{Computer Aided Engineering}
Direct Digital Manufacture, where AM can be used to directly produce
final products, requires Computer Aided Engineering (CAE) tools to evaluate how these parts
would perform prior to AM. This ensures that we can build products right the first time as a
form of Design for Additive Manufacturing (D for AM).

\subsection{Heptic Based CAD}
CAD modeling systems work
in a similar way to Freeform modeling systems to
provide a design environment that is more intuitive than other standard CAD
systems. They often use a robotic haptic feedback device called the Phantom to provide force feedback relating to the virtual modeling environment. An object can
be seen on-screen, but also felt in 3D space using the Phantom. The modeling
environment includes what is known as Virtual Clay that deforms under force
applied using the haptic cursor. This provides a mechanism for direct interaction
with the modeling material, much like how a sculptor interacts with actual clay. Basically, this is CAD for non-engineers for non-engineering applcations.

\section{Classification of AM Processes}
\begin{enumerate}
\item Vat photopolymerization - processes that utilize a liquid photopolymer that is
contained in a vat and processed by selectively delivering energy to cure specific
regions of a part cross-section.
\item Material extrusion - processes that deposit a material by extruding it through a
nozzle, typically while scanning the nozzle in a pattern that produces a part
cross-section.
\item Material jetting: ink-jet printing processes.
\item Binder jetting - processes where a binder is printed into a powder bed in order to
form part cross-sections.
\item Sheet lamination: processes that deposit a layer of material at a time, where the
material is in sheet form.
\item Directed energy deposition - processes that simultaneously deposit a material
(usually powder or wire) and provide energy to process that material through a
single deposition device.
\item Fused Deposition Modeling (FDM).
\end{enumerate}
FDM uses a
heating chamber to liquefy polymer that is fed into the system as a filament. The filament is pushed into the chamber by a tractor wheel arrangement and it is this
pushing that generates the extrusion pressure. In this practical exercise, FDM was used to come up with a 3D object.
