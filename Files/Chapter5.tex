\section{Discussion}
\subsection{3D Scanning}
The reverse engineering process is able to replicate an existing object with a great degree of dimensional accuracy and identical geometry. This eliminates the need for accurate measurements and complicated 3D CAD techniques to obtain a 3D Model of an existing product.\\
While the model obtained is quite close to the original, there are still some inaccuracies in the 3D scan. Some of these are:
\begin{enumerate}
	\item Distorted screw threads.
	\item Blocked holes.
	\item Inaccurate internal geometry.
	\item Creation of non-existent geometry.
\end{enumerate}
Some of these inaccuracies in the 3D Scan can be minimised by using proper technique and setting up a good scanning environment. Some of the ways to improve the quality of a scan are:
\begin{enumerate}
	\item Ensuring the scanning area is well lit
	\item Scanning dark or non-reflective objects. A thin coating such as fine charcoal dust can be applied to the object to improve scan results.
	\item Carrying out several scans to capture geometries on the opposite side of the scanner.
	\item Using a turntable to rotate the specimen at a constant speed, as well as synchronising the angular speed with the scanner's polling rate.
\end{enumerate}