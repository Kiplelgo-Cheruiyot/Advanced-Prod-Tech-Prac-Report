\section{Introduction}
\lhead{\leftmark}
\label{sec:introduction}
\subsection{Additive Manufacturing}
Additive Manufacturing (AM) is a manufacturing technology that uses the additive approach in the fabrication of parts. AM significantly
simplifies the process of producing complex 3D objects directly from CAD data\cite{edgar2015additive}.

3D Printing has become the most commonly used wording to describe AM technologies. This term alludes to the use of a 2D process (printing) and extending them into the third dimension. Significant improvements in accuracy and material properties have seen 3D printing
technology become useful in other applications other than prototyping. These applications are: testing, tooling, manufacturing, etc. 

3D printing is a rapid and seamless process. It also reduces the amount of resources and processes required significantly. With the addition of some supporting technologies like silicone-rubber
molding, drills, polishers, grinders, etc. AM can be possible to manufacture a vast
range of different parts with different characteristics. Workshops which adopt AM
technology can be much cleaner, more streamlined, and more versatile than before\cite{edgar2015additive}.

\subsection{Objectives}
\begin{enumerate}
\item To design a CAD model of a complex 3D part.
\item To fabricate the complex 3D part using an Additive Manufacturing technique (3D Printing)
\end{enumerate}
